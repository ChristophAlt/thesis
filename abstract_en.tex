% $Log: abstract.tex,v $
% Revision 1.1  93/05/14  14:56:25  starflt
% Initial revision
% 
% Revision 1.1  90/05/04  10:41:01  lwvanels
% Initial revision
% 
%
%% The text of your abstract and nothing else (other than comments) goes here.
%% It will be single-spaced and the rest of the text that is supposed to go on
%% the abstract page will be generated by the abstractpage environment.  This
%% file should be \input (not \include 'd) from cover.tex.
In recent years, machine learning emerged as an essential part of many successful applications and businesses. The ever growing amount of data requires these algorithms, many of them sequential in nature, to be executed in a distributed manner on a large scale. This is commonly achieved by exploiting the algorithms stochastic nature to allow for parallel execution at the expense of lowered consistency among the distributed data structures. Even though the quality of the result is not affected, an improper level of consistency can severely affect algorithm performance, resulting in a non optimal convergence rate. Furthermore the level of consistency required to achieve the best performance can change during different periods of algorithm execution. System architecture, topology, algorithm, hardware and data properties influence performance as well. Despite its widespread use, the implications of these aspects of distributed systems on algorithm performance are not yet well understood. This thesis aims to answer the question of how the level of consistency affects the overall performance of distributed machine learning algorithms and also aims to identify strategies for consistency management that can help to mitigate the negative effect on performance. Based on the identified strategies, a protocol will be developed that is able to manage the consistency level of distributed state. The protocol will be designed in a way that allows for dynamic changes in consistency. This enables the possibility of adaptive consistency management in distributed machine learning environments based on heuristics and algorithm progress.
